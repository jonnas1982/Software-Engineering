\documentclass[11pt,a4paper]{article}
\usepackage[utf8]{inputenc}
\usepackage{amsmath}
\usepackage{amsfonts}
\usepackage{amssymb}
\usepackage{graphicx}
\begin{document}
\section{Software process models: waterfall}
\subsection*{Koncept}
\subsubsection*{Software engenering}
Man begyndte at bruge Software engenering, da der var stor kompleksitet ved projekterne og der skete mange fejl.\\
Man kigger på 
\begin{itemize}
\item Specifikation
\item Design og implementering
\item test og validation
\item Evolution – changing the system in response to changing customer needs
\end{itemize}
Software process model: Abstrakt repræsentation af process

\subsubsection*{Waterfall}
Det er en plandriven model, hvilket betyder man starter ud med at ligge hele planen og den følger man så uden at kigge tilbage.\\
Man starter ud med at lave krav, som man så ikke ændre senere, hvor efter man går videre og design fasen, konstruktion, integration, test, installation og så til sidst vedligholdelse.

Man bør bruge den hvis man arbejder med embedded systems, livskritiske systemer eller meget store systemer, da det her kan være svært at lave det samarbejde, som de agile modeller kræver.

Den kræver kun man har kunden indover ved skift mellem phaser.\\
Det er godt hvis ens kontrakt limiter kravene fra starten af.\\
Vi bruger en \textit{Do everything we agreed on} tilgang, da det mindser risikoen for fejl og kunden får alt hvad de har bedt om.\\
Der er meget lidt samarbejde i teamet ud over ved aflevering, da alle opgaver er sat.\\
Prisen er gerne bestemt på forhånd, da man allerede ved hvad der skal ske.
\subsection*{pro/con}
Waterfall er rigtigt godt hvis man har en kontrakt der meget specifikt beskriver hvad der er for nogle krav, som man skal igennem.\\
Den fungere ikke hvis man har et projekt hvor der er konstant skiftende krav, da den ikke tillader man går tilbage og ændre i krav.\\
Korte projekter er ofte lette er overskue fra starten af, så der er ikke nogen grund til at bruge den ekstra energi på en agil model.\\
Man sætter sig ud for hvad ens mål er fra starten af, så alle ved hvor det er man vil hen.\\
Der bliver dokumenteret rigtigt godt, da der er mulighed for at det er forskellige teams, som står for de enkelte steps.\\
Man har sjældent kunde/end user med ind over igennem processen, så der er mulighed for at man ender ud med noget andet end det de vil have, da de sjældent ved hvad det er de vil have.
\subsection*{Andre emner}
Software process model: incremental and iterative\\
Software process model: integration and configuration\\
Kombinering
\newpage
%-------------------------------------------%
\section{Software process model: incremental and iterative}
\subsection*{Koncept}
\subsubsection*{Software engenering}
Man begyndte at bruge Software engenering, da der var stor kompleksitet ved projekterne og der skete mange fejl.\\
Man kigger på 
\begin{itemize}
\item Specifikation
\item Design og implementering
\item test og validation
\item Evolution – changing the system in response to changing customer needs
\end{itemize}
Software process model: Abstrakt repræsentation af process
\subsubsection*{Inkrementiel / Iterativ}
Man kombinere krav, udvikling og validering.\\
Det kan både være plandriven, agilt og oftest et mix.\\
Man udvikler en lille del af produktet (fx UI) får så feedback hvor efter man kan gå videre med den næste del.
\subsection*{pro/con}
Man kan bruge både plan-baseret og agil, hvilket giver mulighed for at bruge den model, som man føler sig mest tryk med.\\
Det er let at få feedback.\\
Der er mulighed for at aflevere brugbar software tidligere, da man hele tiden fokusere på enkelte fonktioner.\\
En manager har brug for at få regulær updates, da det er meget svært at overværer processen.\\
Der kommer let noget rodet koden, som increments bliver tilføjet.

\subsection*{Andre emner}
Software process model: Waterfall\\
Software process model: integration and configuration\\
Kombinering
\newpage

%-------------------------------------------------
\section{Software process model: integration and configuration}
\subsection*{Koncept}
\subsubsection*{Software engenering}
Man begyndte at bruge Software engenering, da der var stor kompleksitet ved projekterne og der skete mange fejl.\\
Man kigger på 
\begin{itemize}
\item Specifikation
\item Design og implementering
\item test og validation
\item Evolution – changing the system in response to changing customer needs
\end{itemize}
Software process model: Abstrakt repræsentation af process

\subsubsection*{Integration and configuration}
Man kigger meget på software, som allerede er blevet udviklet og om man kan genbruge noget af det.\\
Når der bliver kodet noget nyt sørges der for at der er mulighed for at genbruge det ved at configurer det, da det giver mulighed for at sparer tid/penge.
\subsection*{pro/con}
Da man genbruger ting bliver der en mindre risiko for at ting ikke virker, men folk får ikke når så meget en ejer følelse af det hvilket kan sænke kvaliteten af det generelle produkt.\\
Der er en risiko for at man bliver nødt til at gå på kompromi med krav hvis man ønsker at bruge allerede udviklet software, som ikke lever helt op til kravene.
\subsection*{Andre emner}
Software process model: Waterfall\\
Software process model: incremental and iterative\\
\newpage
%----------------------------------------------------
\section{Comparison of plan-driven and agile software engineering processes, including analysis of home grounds}
\subsection{Koncept}
\subsection{pro/con}
\subsection{Andre emner}

\newpage
%----------------------------------------------------
\section{Key features of Scrum}
\subsection*{Koncept}
A framework within which people can address complex adaptive problems, while productively and creatively delivering products of the highest possible value.
\begin{itemize}
\item Roles\\ \textbf{Product Owner}: Ejeren af produktet, derfor ikke en del af teamet, men hjælper med at sætte krav.\\ \textbf{Scrum Master}: Styrer teamet i forhold til hvad der skal ske hvornår, er ikke en del af teamet.\\ \textbf{Team}: Programører, designere mm.
\item \textbf{Sprint Planning}: Planlægning.
\item \textbf{Daily Scrum}: Dagligt møde.
\item \textbf{Sprint Review}: Review afholdt ved afslutningen af hvert sprint.
\item \textbf{Sprint Retrospective}: Møde omkring hvad gik godt, hvad kan gøres bedre og hvad vil vi gøre for at gøre det bedre.
\item \textbf{Backlog refinement}: Holde backlog opdateret.
\item \textbf{Product Burndown/Sprint Burndown}: Completed work per day for det nuværende projekt udgivelse.
\item \textbf{Scrum board}: Fremvisning af backlog
\end{itemize}
Scrum giver \textbf{commitment}, \textbf{courage}, \textbf{focus}, \textbf{openhed} (team) \textbf{respect} (management over for team, derfor tror de på de gør deres arbejde)

Det sker nogen gange følgende \textbf{fejl}: Scrum master implemented as manager who tells team what to do (right way: Facilitator for team), kunde bliver ikke involveret, nye krav og opgaver bliver tilføjet i løbet af et sprint.

\textbf{eXtreame Programing} er godt ved scrum i forhold til kunde-on-site, user stories, planning game.
\subsection*{pro/con}
Effektivt brug af tid og penge\\
Gøre store projekter mere overskulige\\
Godt for hurtigt bevægende projektor\\
Teams for godt overskulighed\\
Adopts feedback\\
Man kan se arbejdet hver person laver\\
Ens scope kan let blive uroligt\\
Hvis folk ikke er committet går det let galt\\
Kan være svært at få igang i store teams\\
Daglige møder iritere folk\\
\subsection*{Andre emner}
\newpage
\section{Key features of RUP}
\subsection{Koncept}
\subsection{pro/con}
\subsection{Andre emner}
\newpage
\section{Requirements Elicitation and Management}
\subsection{Koncept}
\subsection{pro/con}
\subsection{Andre emner}
\newpage
\section{Managing change to requirements}
\subsection{Koncept}
\subsection{pro/con}
\subsection{Andre emner}
\newpage
\section{Quality Control: Verification and Validation}
\subsection{Koncept}
\subsection{pro/con}
\subsection{Andre emner}
\newpage
\section{Risk Management}
\subsection{Koncept}
\subsection{pro/con}
\subsection{Andre emner}
\newpage
\section{Project Planning and Management}
\subsection{Koncept}
\subsection{pro/con}
\subsection{Andre emner}
\newpage
\section{Quality Management: How is quality defined - agile versus plan driven approaches}
\subsection{Koncept}
\subsection{pro/con}
\subsection{Andre emner}
\newpage
\section{Configuration Management}
\subsection{Koncept}
\subsection{pro/con}
\subsection{Andre emner}
\newpage
\end{document}